\begin{abstract}
    In the last years, more and more computers are being created every day, but 
    these computers are not regular smartphones or personal computers, we are 
    talking about lightbulbs, switches, coffee machines, TV, and sensors, if 
    you take any household item you would find a "smart" version of this items, 
    by 2030 it would be more than 25.4 billion of these devices connected to 
    the internet. These devices have grown more from the realm of housing and 
    have reached industries like manufacturing, agriculture, automotive, 
    and energy. 

    Even though there is a market and there is a need for IoT devices the issue 
    that exists right now is fragmentation, any IoT manufacturer uses different 
    standard and that makes it hard for a developer to adopt and integrate into 
    specific platforms, there has been some standardization in the consumer 
    side thanks to Amazon, Google and Apple, the downside of those is that they 
    use proprietary protocols and you need to work within their guidelines 
    which limits the amount of integration that can be performed. 
    
    In this paper, we are proposing an architecture Djinn that enables 
    developers to integrate different IoT devices into a single platform that 
    uses open standards to communicate with the devices while offering the 
    flexibility to integrate with the consumer platform. The main idea of 
    Djinn is to use a device as IoT Gateway, and from this gateway, 
    developers have the freedom to create plugins that talk to the specific 
    protocol of each IoT device, then the gateway communicates to an external 
    server that as long it implements an API can be integrated into the 
    customer platform.
\end{abstract}