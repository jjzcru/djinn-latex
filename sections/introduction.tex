\section{Introduction}
The term IoT was coined by Kevin Ashton in 1999, since that time an entire 
market as evolved around that concept, this types of devices are mostly known 
thanks to platforms like HomeKit, Google Assistant and Alexa which democratized 
the usage of this devices in the consumer market, this convenience brings two 
things. Centralization which makes user dependent on this companies to make 
their devices work and Standardization which enables developers transfer 
knowledge from the implementation of one device into another. But there is a 
flip side of this, we can achieve standardization without the centralization, 
centralization is only achieve when we use proprietary protocols we can achieve 
standardization using open protocols, this scenario is what the internet looked 
like years ago and we ended using TCP/IP and HTTP as the standard for the web.

Right now we still do not have a definitive answer on THE protocol for IoT but 
there is a set of open standards that we can use, some of the are already being
 used in "normal" computing like Bluetooth and HTTP and others are more 
 specialized for IoT like CoAP, Zigbee, Z-Wave, MQTT, and some are currently 
 being develop to tackle this fragmentation issue like Matter.

Regardless of the communication protocol the architecture that is proposed in 
this paper uses device as a gateway that standardize the behavior of the 
devices, either sensors or actuators, and use a plugin architecture that enable 
the providers, or third-party developers, to develop integration for different 
devices, regardless of the protocol.

The developers of this plugins focus mostly on how to talk to each particular 
device and then this devices get standardized in  the platform, this enables 
developers to think about the devices from a high level instead of worrying 
about the implementation detail of each IoT device.

The architecture uses a computer that lives in the same network that the IoT 
devices and serve the purpose of an IoT Gateway, this computer also enables 
user to communicate with the IoT devices without the need for an internet 
connection. 

The functionalities can be extended by adding a cloud integration, the IoT 
Gateway connects to a remote cloud server via an interface which can be 
implemented in any language, and sends information of the devices and also is 
able to listen remote commands from the cloud, there isn't any requirement on 
how this interface is implemented which gives developer flexibility to add 
integration to another platform at this level.

The architecture on the board focus mostly on data messaging and not data 
processing or state management, it's main goal is to read states of the devices 
and send command to them which makes it more effective to have a small machine 
as an IoT Gateway and offset all the resource intensive task to the cloud, the 
main goal of the board is to request the state of the devices and send command 
to those devices.

The cloud integration would be more suited for task like data-processing, 
machine learning training, platform integration and automation and resource 
management. In a cloud environment is more easy to add CPU, Memory or Storage 
than the constraint environment like the board.

The rest of the paper is organized as follows: Section 2 is going to review the 
existing protocols that are used in the industry, Section 3 is going to 
describe the architecture Section 4 is going to be a proposal of the resource 
definition to standardize the IoT devices and Section 5 shows some use cases 
for this architecture and Section 6 discuss some perspective on how this 
architecture can be extended.
