\section{Background}
{\it Introduction for the background}

\subsection{Network Protocols}
One of the goal of IoT is to have multiple of them to sense the world (sensors) 
and change the state of the world as well (actuators) but this sensor are limit 
to a single functionality so we need different devices that can act upon 
different kind of inputs.

The same way humans have different electrical terminals to sense the world, 
temperature, touch, texture and have multiple ways to change the world upon 
those inputs, breathing, muscle movement, temperature changes and the brain 
is the organ responsible of this coordination, in IoT we need the same, we need 
a network of different devices to sense the physical world and actuators to be 
able change the state of that world. 

The same way humans have different protocols to talk to the brain called 
neurotransmitters, in IoT there is also a set of protocols that are used to 
coordinate the communication from and to the IoT devices, the main advantage 
that we have in IoT is that amount of protocols is not static as humans and 
new protocols are being created to add more functionalities, like 
Matter \cite{matter}. 

\subsubsection{HTTP}
HTTP is the protocol of the internet, this protocol power all the websites 
that you see in your browser and also provide the entry point for API 
integration that is offered by other companies. This protocol uses the 
request-response paradigm, which a machine starts a request and a server 
reply, this paradigm is easy to implement in IoT because of it's story and 
support but it also has it's drawbacks, because this protocol is 
half-duplex so the information can only flow one direction at a time and 
every time a request is made it requires to perform a three-way handshake 
which is bottleneck if we are sending a lot of information in a short 
period of time like a real time system.

This protocol is good enough if we are tracking sporadic events like an on-off 
switch but it's a problem if we required to track real time information.

\subsubsection{CoAP}
CoAP is a web transfer protocol, specified in IETF RFC 7252, which was 
designed for devices with limited computation and that also reside in a 
network with limited bandwidth which is the most common scenario for 
IoT Devices.

This protocol is a lightweight version of the HTTP protocol specially 
designed for IoT because it support multicast, it has a low overhead, 
uses UDP instead of TCP and is simple because it uses the same REST 
interaction model as HTTP.

CoAP also requires DTLS for security, which is the TLS over the UDP protocol, 
any other control access or security is performed at the application level with 
provides developers the flexibility to tailor the security model to 
their own needs.

\subsubsection{MQTT}
MQTT is another protocol that is perfectly suited for IoT Devices, it's 
lightweight, it has lower power and bandwidth consumption and also has low 
latency instead of focusing a request-response paradigm like HTTP or CoAP 
is based in the publish-subscribe model.

The benefit of said model is that is easy to scale because you don't need to 
worry about what is the network topology, the publishers do not need to worry 
about where the subscribers are located and the subscribers do not need to 
worry about who the publisher is, the only thing that both requires is to 
know where is the MQTT broker.

With this kind of flexibility, simplicity, scalability, low overhead and low 
bandwidth consumption this protocol would be the ideal protocol for IoT 
devices that required to send data in real time.

\subsubsection{Bluetooth}
Bluetooth provides a short distance wireless communication with a lower power 
consumption than protocols that uses WiFi, like HTTP, CoAP, MQTT, for IoT is 
more common to use Bluetooth Low Energy (BLE) because it reduces more the 
power consumption, the downside of this protocol is that the data transfer
 are limited at 1 Mbps and the range is also smaller compared to WiFi 
 protocols (100 meters).

Because Bluetooth is a Personal Area Network technology (PAN) once the 
communication is establish it provides a more reliable point-to-point 
communication that the WiFi based protocols, because it doesn't depend on a 
router for device discovery, since Bluetooth is more power efficient than 
WiFi this protocol is suited for those devices that are mission critical that 
do not required auto-discovery and that are required to run on battery for 
long periods of time.

\subsection{Frameworks}
There are existing frameworks that have been develop to address the issue of 
managing IoT devices from heterogenous vendors, some of them are all in one 
solutions and other are an abstract implementation that provides a set of 
principles on how the framework should be develop without imposing any 
implementation detail.

\subsubsection{OCF Framework}

The OCF framework was created by the Open Connectivity Foundation, this 
framework consists on three layers.

\paragraph{Transports}
Refers to the plugin interface that supports multiple number of transport 
plugin modules which includes IPv4, IPv6, Ethernet, Wifi and Bluetooth LE but 
the interface is transports agnostic.

\paragraph{Core Framework}
This is the centerpiece of the OCF architecture, it describes a resource 
abstraction which is a set of tag-value pairs that describe the devices 
using JSON, the reason for this is that all the aspects of the network 
could be describe declaratively while supporting CRUDN semantics.

\paragraph{Profiles}
Profiles are libraries of resources that have a common functionality, which 
are group according to target deployment, this kinds of profiles are 
extensible, the data model of OCF supports resource introspection, which 
enable the inspection of the device by a client.

\subsubsection*{OneM2M}