\documentclass[sigconf]{acmart}


%% \AtBeginDocument{%
  %% \providecommand\BibTeX{{%
    %% \normalfont B\kern-0.5em{\scshape i\kern-0.25em b}\kern-0.8em\TeX}}}

%% \setcopyright{acmcopyright}
%% \copyrightyear{2018}
%% \acmYear{2018}
%% \acmDOI{XXXXXXX.XXXXXXX}

%% These commands are for a PROCEEDINGS abstract or paper.
%% \acmConference[IoT Conference 2022]{IoT Conference 2022}{August 29, 2022} {Woodstock, NY}
%
%  Uncomment \acmBooktitle if th title of the proceedings is different
%  from ``Proceedings of ...''!
%
%\acmBooktitle{Woodstock '18: ACM Symposium on Neural Gaze Detection,
%  June 03--05, 2018, Woodstock, NY} 
%\acmPrice{15.00}
%\acmISBN{978-1-4503-XXXX-X/18/06}


%%
%% Submission ID.
%% Use this when submitting an article to a sponsored event. You'll
%% receive a unique submission ID from the organizers
%% of the event, and this ID should be used as the parameter to this command.
%%\acmSubmissionID{123-A56-BU3}

%%
%% For managing citations, it is recommended to use bibliography
%% files in BibTeX format.
%%
%% You can then either use BibTeX with the ACM-Reference-Format style,
%% or BibLaTeX with the acmnumeric or acmauthoryear sytles, that include
%% support for advanced citation of software artefact from the
%% biblatex-software package, also separately available on CTAN.
%%
%% Look at the sample-*-biblatex.tex files for templates showcasing
%% the biblatex styles.
%%

%%
%% The majority of ACM publications use numbered citations and
%% references.  The command \citestyle{authoryear} switches to the
%% "author year" style.
%%
%% If you are preparing content for an event
%% sponsored by ACM SIGGRAPH, you must use the "author year" style of
%% citations and references.
%% Uncommenting
%% the next command will enable that style.
%%\citestyle{acmauthoryear}

%%
%% end of the preamble, start of the body of the document source.
\begin{document}

%%
%% The "title" command has an optional parameter,
%% allowing the author to define a "short title" to be used in page headers.
\title{Djinn: A framework for IoT and the Cloud}

%%
%% The "author" command and its associated commands are used to define
%% the authors and their affiliations.
%% Of note is the shared affiliation of the first two authors, and the
%% "authornote" and "authornotemark" commands
%% used to denote shared contribution to the research.

\author{José J. Cruz Torres}
\affiliation{%
  \institution{Stevens Institute of Technology}
  \city{Hoboken}
  \country{USA}}
\email{jcruz11@stevens.edu}

\author{Ying Wang}
\affiliation{%
  \institution{Stevens Institute of Technology}
  \city{Hoboken}
  \country{USA}}
\email{ywang6@stevens.edu}

%%
%% By default, the full list of authors will be used in the page
%% headers. Often, this list is too long, and will overlap
%% other information printed in the page headers. This command allows
%% the author to define a more concise list
%% of authors' names for this purpose.
\renewcommand{\shortauthors}{Cruz and Wang, et al.}

%%
%% The abstract is a short summary of the work to be presented in the
%% article.
\begin{abstract}
    In the last years, more and more computers are being created every day, but 
    these computers are not regular smartphones or personal computers, we are 
    talking about lightbulbs, switches, coffee machines, TV, and sensors, if 
    you take any household item you would find a "smart" version of this items, 
    by 2030 it would be more than 25.4 billion of these devices connected to 
    the internet. These devices have grown more from the realm of housing and 
    have reached industries like manufacturing, agriculture, automotive, 
    and energy. 

    Even though there is a market and there is a need for IoT devices the issue 
    that exists right now is fragmentation, any IoT manufacturer uses different 
    standard and that makes it hard for a developer to adopt and integrate into 
    specific platforms, there has been some standardization in the consumer 
    side thanks to Amazon, Google and Apple, the downside of those is that they 
    use proprietary protocols and you need to work within their guidelines 
    which limits the amount of integration that can be performed. 
    
    In this paper, we are proposing an architecture Djinn that enables 
    developers to integrate different IoT devices into a single platform that 
    uses open standards to communicate with the devices while offering the 
    flexibility to integrate with the consumer platform. The main idea of 
    Djinn is to use a device as IoT Gateway, and from this gateway, 
    developers have the freedom to create plugins that talk to the specific 
    protocol of each IoT device, then the gateway communicates to an external 
    server that as long it implements an API can be integrated into the 
    customer platform.
\end{abstract}

%%
%% Keywords. The author(s) should pick words that accurately describe
%% the work being presented. Separate the keywords with commas.
\keywords{iot, architecture}

%%
%% This command processes the author and affiliation and title
%% information and builds the first part of the formatted document.
\maketitle
\section{Introduction}
The term IoT was coined by Kevin Ashton in 1999, since that time an entire 
market as evolved around that concept, this types of devices are mostly known 
thanks to platforms like HomeKit, Google Assistant and Alexa which democratized 
the usage of this devices in the consumer market, this convenience brings two 
things. Centralization which makes user dependent on this companies to make 
their devices work and Standardization which enables developers transfer 
knowledge from the implementation of one device into another. But there is a 
flip side of this, we can achieve standardization without the centralization, 
centralization is only achieve when we use proprietary protocols we can achieve 
standardization using open protocols, this scenario is what the internet looked 
like years ago and we ended using TCP/IP and HTTP as the standard for the web.

Right now we still do not have a definitive answer on THE protocol for IoT but 
there is a set of open standards that we can use, some of the are already being
 used in "normal" computing like Bluetooth and HTTP and others are more 
 specialized for IoT like CoAP, Zigbee, Z-Wave, MQTT, and some are currently 
 being develop to tackle this fragmentation issue like Matter.

Regardless of the communication protocol the architecture that is proposed in 
this paper uses device as a gateway that standardize the behavior of the 
devices, either sensors or actuators, and use a plugin architecture that enable 
the providers, or third-party developers, to develop integration for different 
devices, regardless of the protocol.

The developers of this plugins focus mostly on how to talk to each particular 
device and then this devices get standardized in  the platform, this enables 
developers to think about the devices from a high level instead of worrying 
about the implementation detail of each IoT device.

The architecture uses a computer that lives in the same network that the IoT 
devices and serve the purpose of an IoT Gateway, this computer also enables 
user to communicate with the IoT devices without the need for an internet 
connection. 

The functionalities can be extended by adding a cloud integration, the IoT 
Gateway connects to a remote cloud server via an interface which can be 
implemented in any language, and sends information of the devices and also is 
able to listen remote commands from the cloud, there isn't any requirement on 
how this interface is implemented which gives developer flexibility to add 
integration to another platform at this level.

The architecture on the board focus mostly on data messaging and not data 
processing or state management, it's main goal is to read states of the devices 
and send command to them which makes it more effective to have a small machine 
as an IoT Gateway and offset all the resource intensive task to the cloud, the 
main goal of the board is to request the state of the devices and send command 
to those devices.

The cloud integration would be more suited for task like data-processing, 
machine learning training, platform integration and automation and resource 
management. In a cloud environment is more easy to add CPU, Memory or Storage 
than the constraint environment like the board.

The rest of the paper is organized as follows: Section 2 is going to review the 
existing protocols that are used in the industry, Section 3 is going to 
describe the architecture Section 4 is going to be a proposal of the resource 
definition to standardize the IoT devices and Section 5 shows some use cases 
for this architecture and Section 6 discuss some perspective on how this 
architecture can be extended.

\section{Background}
{\it Introduction for the background}

\subsection{Network Protocols}
One of the goal of IoT is to have multiple of them to sense the world (sensors) 
and change the state of the world as well (actuators) but this sensor are limit 
to a single functionality so we need different devices that can act upon 
different kind of inputs.

The same way humans have different electrical terminals to sense the world, 
temperature, touch, texture and have multiple ways to change the world upon 
those inputs, breathing, muscle movement, temperature changes and the brain 
is the organ responsible of this coordination, in IoT we need the same, we need 
a network of different devices to sense the physical world and actuators to be 
able change the state of that world. 

The same way humans have different protocols to talk to the brain called 
neurotransmitters, in IoT there is also a set of protocols that are used to 
coordinate the communication from and to the IoT devices, the main advantage 
that we have in IoT is that amount of protocols is not static as humans and 
new protocols are being created to add more functionalities, like 
Matter \cite{matter}. 

\subsubsection{HTTP}
HTTP is the protocol of the internet, this protocol power all the websites 
that you see in your browser and also provide the entry point for API 
integration that is offered by other companies. This protocol uses the 
request-response paradigm, which a machine starts a request and a server 
reply, this paradigm is easy to implement in IoT because of it's story and 
support but it also has it's drawbacks, because this protocol is 
half-duplex so the information can only flow one direction at a time and 
every time a request is made it requires to perform a three-way handshake 
which is bottleneck if we are sending a lot of information in a short 
period of time like a real time system.

This protocol is good enough if we are tracking sporadic events like an on-off 
switch but it's a problem if we required to track real time information.

\subsubsection{CoAP}
CoAP is a web transfer protocol, specified in IETF RFC 7252, which was 
designed for devices with limited computation and that also reside in a 
network with limited bandwidth which is the most common scenario for 
IoT Devices.

This protocol is a lightweight version of the HTTP protocol specially 
designed for IoT because it support multicast, it has a low overhead, 
uses UDP instead of TCP and is simple because it uses the same REST 
interaction model as HTTP.

CoAP also requires DTLS for security, which is the TLS over the UDP protocol, 
any other control access or security is performed at the application level with 
provides developers the flexibility to tailor the security model to 
their own needs.

\subsubsection{MQTT}
MQTT is another protocol that is perfectly suited for IoT Devices, it's 
lightweight, it has lower power and bandwidth consumption and also has low 
latency instead of focusing a request-response paradigm like HTTP or CoAP 
is based in the publish-subscribe model.

The benefit of said model is that is easy to scale because you don't need to 
worry about what is the network topology, the publishers do not need to worry 
about where the subscribers are located and the subscribers do not need to 
worry about who the publisher is, the only thing that both requires is to 
know where is the MQTT broker.

With this kind of flexibility, simplicity, scalability, low overhead and low 
bandwidth consumption this protocol would be the ideal protocol for IoT 
devices that required to send data in real time.

\subsubsection{Bluetooth}
Bluetooth provides a short distance wireless communication with a lower power 
consumption than protocols that uses WiFi, like HTTP, CoAP, MQTT, for IoT is 
more common to use Bluetooth Low Energy (BLE) because it reduces more the 
power consumption, the downside of this protocol is that the data transfer
 are limited at 1 Mbps and the range is also smaller compared to WiFi 
 protocols (100 meters).

Because Bluetooth is a Personal Area Network technology (PAN) once the 
communication is establish it provides a more reliable point-to-point 
communication that the WiFi based protocols, because it doesn't depend on a 
router for device discovery, since Bluetooth is more power efficient than 
WiFi this protocol is suited for those devices that are mission critical that 
do not required auto-discovery and that are required to run on battery for 
long periods of time.

\subsection{Frameworks}
There are existing frameworks that have been develop to address the issue of 
managing IoT devices from heterogenous vendors, some of them are all in one 
solutions and other are an abstract implementation that provides a set of 
principles on how the framework should be develop without imposing any 
implementation detail.

\subsubsection{OCF Framework}

The OCF framework was created by the Open Connectivity Foundation, this 
framework consists on three layers.

\paragraph{Transports}
Refers to the plugin interface that supports multiple number of transport 
plugin modules which includes IPv4, IPv6, Ethernet, Wifi and Bluetooth LE but 
the interface is transports agnostic.

\paragraph{Core Framework}
This is the centerpiece of the OCF architecture, it describes a resource 
abstraction which is a set of tag-value pairs that describe the devices 
using JSON, the reason for this is that all the aspects of the network 
could be describe declaratively while supporting CRUDN semantics.

\paragraph{Profiles}
Profiles are libraries of resources that have a common functionality, which 
are group according to target deployment, this kinds of profiles are 
extensible, the data model of OCF supports resource introspection, which 
enable the inspection of the device by a client.

\subsubsection*{OneM2M}
\input{./sections/architecture.tex}
\input{./sections/resource-definition.tex}
\input{./sections/use-cases.tex}
\input{./sections/future-work.tex}

%%
%% The acknowledgments section is defined using the "acks" environment
%% (and NOT an unnumbered section). This ensures the proper
%% identification of the section in the article metadata, and the
%% consistent spelling of the heading.
\begin{acks}
To Robert, for the bagels and explaining CMYK and color spaces.
\end{acks}

%%
%% The next two lines define the bibliography style to be used, and
%% the bibliography file.
\bibliographystyle{ACM-Reference-Format}
\bibliography{sample-base}

%%
%% If your work has an appendix, this is the place to put it.
\appendix

\section{Research Methods}

\subsection{Part One}

Lorem ipsum dolor sit amet, consectetur adipiscing elit. Morbi
malesuada, quam in pulvinar varius, metus nunc fermentum urna, id
sollicitudin purus odio sit amet enim. Aliquam ullamcorper eu ipsum
vel mollis. Curabitur quis dictum nisl. Phasellus vel semper risus, et
lacinia dolor. Integer ultricies commodo sem nec semper.

\subsection{Part Two}

Etiam commodo feugiat nisl pulvinar pellentesque. Etiam auctor sodales
ligula, non varius nibh pulvinar semper. Suspendisse nec lectus non
ipsum convallis congue hendrerit vitae sapien. Donec at laoreet
eros. Vivamus non purus placerat, scelerisque diam eu, cursus
ante. Etiam aliquam tortor auctor efficitur mattis.

\section{Online Resources}

Nam id fermentum dui. Suspendisse sagittis tortor a nulla mollis, in
pulvinar ex pretium. Sed interdum orci quis metus euismod, et sagittis
enim maximus. Vestibulum gravida massa ut felis suscipit
congue. Quisque mattis elit a risus ultrices commodo venenatis eget
dui. Etiam sagittis eleifend elementum.

Nam interdum magna at lectus dignissim, ac dignissim lorem
rhoncus. Maecenas eu arcu ac neque placerat aliquam. Nunc pulvinar
massa et mattis lacinia.

\bibliographystyle{unsrt}
\bibliography{references} 

\end{document}
\endinput
%%
%% End of file `sample-sigconf.tex'.
